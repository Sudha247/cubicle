% /nfs/pdx/proj/dt/scl03/fpv/agoel3/tools/cubicle

\documentclass{llncs}
%\usepackage[utf8]{inputenc}
\usepackage{color}
\usepackage{verbatim}
\usepackage{alltt}
\usepackage{xspace}
\usepackage{amsmath}
\usepackage{amsfonts}
\usepackage{amssymb}

\pagestyle{plain}

\title{Cubicle: A Parallel SMT-based Model Checker for Parameterized
Systems}
\subtitle{Tool Paper}

\author{Sylvain Conchon$^1$ \and Amit Goel$^2$ \and Sava Krsti\'c$^2$
%  \\ Alain Mebsout$^1$ \and Fatiha Za\"{\i}di$^1$}
  \\ Alain Mebsout$^1$ \and Fatiha Za\"{i}di$^1$}
\institute{$^1$LRI, Universit\'e Paris Sud\\$^2$Strategic CAD Labs, Intel
  Corporation}

\definecolor{green}{rgb}{0.4,0.4,0.4}
%\newcommand{\trcolor}[1]{\textcolor{blue}{#1}}
\newcommand{\trcolor}[1]{\textcolor{black}{#1}}
%\newcommand{\guard}[1]{\textcolor{red}{#1}}
\newcommand{\guard}[1]{\textcolor{black}{#1}}
\newcommand{\affect}[1]{\textcolor{green}{#1}}

\definecolor{lightred}{rgb}{1.0,0.3,0.3}
\definecolor{darkred}{rgb}{0.4,0.0,0.0}
% Command for inserting a todo item
\newcommand{\todo}[1]{%
  \marginpar{%
      \fcolorbox{darkred}{lightred}{%
        \parbox{\marginparwidth}{\scriptsize \textbf{[TODO]} \small #1}%
      }%
 }%
}%

\def\figref#1{Figure~\ref{#1}}
\def\MCMT{\textsc{mcmt}\xspace}
\def\marginal#1{\marginpar{\scriptsize{#1}}} 
\def\pre{\mathit{pre}}
%\def\VV{\mathcal{V}}
\def\VV{{V}}
%\def\QQ{\mathcal{Q}}
\def\QQ{{Q}}
\def\eqdef{\triangleq}

\begin{document}

\maketitle

\begin{abstract}
  Cubicle is a new model checker for verifying safety properties of
  parameterized systems. It implements a parallel symbolic backward reachability
  procedure using Satisfiabilty Modulo Theories.  Experiments done on classic
  and challenging mutual exclusion algorithms and cache coherence protocols show
  that Cubicle is effective and competitive with state-of-the-art model
  checkers.
\end{abstract}

\section{Tool Overview}

Cubicle is used to verify safety properties of \emph{array-based systems}. This
is a syntactically restricted class of para�metrized transition systems with
states represented as arrays indexed by an arbitrary number of
processes~\cite{GhilardiNRZ08}. Cache coherence protocols and mutual exclusion
algorithms are typical examples of such systems. Cubicle model-checks by a
symbolic backward reachability analysis on infinite sets of states represented
by specific simple formulas, called \emph{cubes}.

%, that are existentially quantified conjunctions of predicates.

Cubicle is an open source software based on theoretical work in
\cite{AbdullaCJT96} and \cite{mcmt-foundations}. It is inspired by and
closely related to the model checker MCMT~\cite{mcmt}, from which, in
addition to revealing the implementation details, it mainly differs in
a more friendly input language and concurrent architecture.

%It consists in repeatedly computing the preimages of cubes, and discharging
%safety and fixpoint checks by an SMT solver.  The tool takes advantage of the
%symbolic representation of states as cubes (by finding symmetries, inclusions
%etc.), as well as a tightly integrated ad-hoc SMT solver. Its implementation
%also exploits multi-core architectures.

Cubicle is written in OCaml. Its SMT solver is a tightly integrated, lightweight
and enhanced version of Alt-Ergo~\cite{ccx}; and its parallel implementation
relies on the Functory library~\cite{functory}. Cubicle is available at
\verb|http://cubicle.lri.fr|.

%Cubicle is written in OCaml. Its SMT solver is a lightweight and enhanced
%version of Alt-Ergo~\cite{ccx} and its parallel implementation relies on the
%Functory library~\cite{functory}. Cubicle is available at
%\verb|http://cubicle.lri.fr|.

% (2600 \emph{loc.}  for the model checker)
%  and 4500 \emph{loc.}  for the SMT solver)
% implemented in the model checker MCMT~\cite{mcmt}.



\section{System Description Language}

Cubicle's input language is a typed version of Mur$\varphi$~\cite{murphi},
rudimentary at the moment, but more user-friendly than \MCMT and sufficiently
expressive for typical parameterized systems. A system is described in Cubicle
by: (1) a set of type, variable, and array declarations; (2) a formula for the
initial states; and (3) a set of transitions. \figref{fig-dekk} shows the
language on an example system \texttt{Mutex}.


Systems are parametrized by a set of \emph{process identifiers}, denoted by the
built-in type \texttt{proc}.  Standard types \texttt{int}, \texttt{real}, and
\texttt{bool} are also built in. Additionally, the user can specify abstract
types and enumerations with simple declarations like ``\texttt {type data}'' and
``\texttt{type msg = Empty | Req | Ack}''.
%

\begin{figure}[ht]
\begin{minipage}[b]{0.5\linewidth}
\centering
\begin{alltt}
{\bf{var}} Turn : proc
{\bf{array}} Want[proc] : bool
{\bf{array}} Crit[proc] : bool

{\bf{init}} (z) \{ 
  Want[z] = False && Crit[z] = False 
\}

{\bf{unsafe}} (x y) \{ 
  Crit[x] = True && Crit[y] = True 
\}

{\bf{transition}} req (i)
{\bf{requires}} \{ Want[i] = False \}
\{ Want[j] := case 
           | i = j : True
           | _ : Want[j]  \}
\end{alltt}
%
\end{minipage}
\hspace{0.5cm}
\begin{minipage}[b]{0.5\linewidth}
\centering
\begin{alltt}
{\bf{transition}} enter (i)
{\bf{requires}} \{ Want[i] = True 
        && Crit[i] = False 
        && Turn = i \}
\{ Crit[j] := case
           | i = j : True
           | _ : Crit[j]  \}

{\bf{transition}} exit (i)
{\bf{requires}} \{ Crit[i] = True \}
\{ Turn := . ;
  Crit[j] := case
           | i = j : False 
           | _ : Crit[j]  ;
  Want[j] := case
           | i = j : False
           | _ : Want[j]  \}
\end{alltt}
%
\end{minipage}
\caption{Complete Cubicle code of \texttt{Mutex}, a simple mutual exclusion algrorithm.}
\label{fig-dekk}
\end{figure}


The system's state is defined by a set of global variables and
\texttt{proc}-indexed arrays. The initial states are defined by a universal
conjunction of literals characterizing the values for some variables and array
entries. A state of our example system \texttt{Mutex} consists of a process
identifier \texttt{Turn} and two boolean arrays \texttt{Want} and \texttt{Crit};
a state is initial iff both arrays are constantly \texttt{false}.

Transitions are given in the usual guard/action form and may be
parameterized by (one or more) process identifiers. They define the
system's execution: an infinite loop that at each iteration: (1)
non-deterministically chooses a transition instance whose guard is
true in the current state; and (2) updates state variables according
to the action of the fired transition instance.  Guards must be of the
form $F \land \forall\bar x.(\Delta \Rightarrow F')$, where $F,F'$
are conjunctions of literals (equations, disequations or inequations),
and $\Delta$ says that every $\bar x$-variable is distinct from every
parameter of the transition.  Assignments can be non-deterministic, as
in ``\texttt{Turn~:=~.}'' in transition \texttt{exit} in
\texttt{Mutex}. Array updates are coded by a case construct where each
condition is a conjunction of literals, and the default case is
denoted by \verb|_|.

%These formulas must be in disjunctive normal form and
%universally quantified by processes different from the transition's arguments.
%\marginal{Clarify} 



The safety property to be verified is expressed in its negated form as
a formula that represents unsafe states. Each unsafe formula must be a
\emph{cube}, i.e., have the form $\exists \bar x. (\Delta \land F)$,
where $\Delta$ is the conjunction of all disequations between the
variables in $\bar x$, and $F$ is a conjunction of literals. In the
code, we leave the $\Delta$ part implicit. Thus in \texttt{Mutex}, the
unsafe states are those in which \texttt{Crit[x]} and \texttt{Crit[y]}
are true for two distinct process identifiers \texttt{x,y}.


\section{Implementation Details and Optimizations}
\label{sec:optims}



For a state formula $\Phi$ and a transition instance $t$, let
$\pre_t(\Phi)$ be the formula describing the set of states from which
a $\Phi$-state can be reached in one $t$-step. Let also $\pre(\Phi)$
be the union of $\pre_t(\Phi)$ for all possible $t$. In its simplest
form, the backward reachability algorithm constructs a sequence
$\Phi_0,\Phi_1,\ldots$ such that $\Phi_0$ is the system's unsafe
condition and $\Phi_{i+1} = \Phi_i \lor \pre(\Phi_{i})$. The algorithm
terminates with the first $\Phi_n$ that fails the \emph{safety check}
(consistency with the initial condition), or passes the \emph{fixpoint
  check} $\Phi_{n}\vdash\Phi_{n-1}$.

In array-based systems, $\pre_t(\phi)$ can be representend as a union
(disjunction) of cubes, for every cube $\phi$ and every $t$. Thus, the
$\Phi_i$ above are unions of cubes too, and the algorithm above can be
modified to work only with cubes, as follows.  Maintain a set $\VV$
and a priority queue $\QQ$ of \emph{visited} and \emph{unvisited
  cubes} respectively. Initially, let $\VV$ be empty and let $\QQ$
contain the system's unsafe condition. Then, at each iteration, take
the highest-priority cube $\phi$ from $\QQ$ and do the safety check
for it, same as the above. If it fails, terminate with ``system
unsafe''.  If the safety check passes, proceed to the
\emph{subsumption check} $\phi \vdash \bigvee_{\psi\in\VV}\psi$. If
this fails, then add $\phi$ to $\VV$, compute all cubes in
$\pre_t(\phi)$ (for every $t$), add them to $\QQ$, and move on to the
next iteration. If the subsumption check succeeds, then drop $\phi$
from consideration and move on. The algorithm terminates when a safety
check fails or $\QQ$ becomes empty.  When an unsafe cube is found,
Cubicle actually produces a counterexample trace.



Safety checks, being ground satisfiablity queries, are easy for SMT
solvers. The challenge is in subsumption checks $\phi
\vdash\bigvee_{\psi\in\VV}\psi$ because of their size and the
``existential implies existential'' logical form. Assuming $\phi
\eqdef \exists\bar x.F$ and $\psi \eqdef \exists\bar y. G_\psi$
($\psi\in\VV$), the subsumption check translates into the validity
check for the ground formula $H \eqdef (F \Rightarrow
\bigvee_{\psi\in\VV}\bigvee_{\sigma\in\Sigma}(G_\psi)\sigma)$, where
$\Sigma$ is the set of all substitutions from $\bar y$ to $\bar
x$. Now, viewing any cube $G_\psi\sigma$ as a set of literals, one can
make two useful comparisons with $F$: (1) if $G_\psi\sigma$ is a
subset of $F$, then $H$ is valid; (2) if $G_\psi\sigma$ contains a
literal that directly contradicts a literal of $F$, then
$G_\psi\sigma$ is redundant in $H$ (can be removed without logically
changing $H$). Cubicle aggressively attempts to prove $H$ by building
and verifying it incrementally, adding one disjunct to its consequent
at a time. Essentially, it examines all pairs $(\psi,\sigma)$
one-by-one, stopping the process when the current overapproximation of
$H$ becomes known to be valid.  For each pair $(\psi,\sigma)$, the
cube $G_\psi\sigma$ is first checked for redundancy; if redundant, it
is ignored and a new pair $(\psi,\sigma)$ is processed. If not
redundant, the cube is subject to the subset check for $F \vdash
G_\psi\sigma$. If this check succeeds, $H$ is claimed valid; otherwise
$G_\psi\sigma$ gets added to $H$ (as a disjunct of its consequent) and
the SMT solver checks if the newly obtained (weakened) $H$ becomes
valid.



Cubicle's integration with the SMT solver at the API level is crucial for
efficient treatment of the subsumption check. For any such check, a single
context for the SMT solver is used; it just gets incremented and repeatedly
verified.  To support the efficient (symmetry-reduced) and exhaustive
application of the inexpensive redundancy and subset checks, cubes are
maintained in normal form where variables are renamed and implied literals
removed at construction time.



The strategy for exploring the cube space is also essential. It pays to visit as
few cubes as possible, which suggess giving priority to more ``generic'' cubes
(those that represent larger sets of states).  Thus, neither breadth-first nor
depth-first search are good in their pure form.  By default, Cubicle uses BFS
(changeable with the \texttt{-search} option to DFS or some variants) combined
with a heuristically delayed treatment of some cubes.  Currently, a cube is
delayed if it introduces new process variables or does not contribute new
information on arrays. Finally, Cubicle can remove cubes from $\mathcal{V}$ when
they become subsumed by a new cube.

Following \MCMT, Cubicle supports user-supplied invariants and invariant
synthesis, both of which can significantly reduce the search.  Subsets of
visited nodes that only contain predicates over a unique process variable are
used as candidate invariants. Each of them is verified by starting a new
resource limited backward reachability analysis. Cubicle can also discover
``subtyping invariants'' (saying that a variable can take only a selected subset
of values) by a static analysis and these invariants can be natively exploited
by the SMT solver which supports definitions of subtypes for enumerated
data-types.


\section{Multi-Core Architecture}

A natural way to scale up model checkers is to parallelize their CPU
intensive tasks to take advantage of the widespread availability of
multi-core machines or clusters~\cite{GrumbergHIS05,divine,eddy}.  In
our framework, this is achieved by parallelizing the backward
reachability loop and the generation of invariants. As mentioned
above, since invariant synthesis is done independently from the main
loop, it is straightforward to do it in parallel. However, concerning
the loop itself, a naive parallel implementation would lose the
precise guidance of the exploration\footnote{Our experiments showed
  that a non-deterministic parallel exploration can be worse than a
  guided sequential search.}, and more importantly, could break the
correctness of the tool because of an unsafe use of some optimizations
described in the previous section.

In our setting, the most resource consuming tasks are fixpoints checks
which can be hard problems even for efficient SMT solvers. To gain
efficiency, we implemented a concurrent version of BFS based on the
observation that all such computations arising at the same level of
the search tree can be executed in parallel.  Our implementation is
based on a centralized master/workers architecture.  The master
assigns fixpoints to workers and a synchronization barrier is placed
at each level of the tree to retain a BFS order. The master
asynchronously computes the preimages of nodes that are not verified
as fixpoints by the workers. In the meanwhile, the master can also
assign invariant generation tasks that will be processed by available
workers. Finally, to safely delete nodes from $\mathcal{V}$, the
master must discard the results about nodes that have been deleted
while they were being checked by a worker.

Cubicle provides a concurrent breadth-first exploration of the search
space using $n$ parallel processes on a multi-core machine with the
\texttt{-j} $n$ option. The implementation is based on
Functory~\cite{functory}, an OCaml library with a rich functional
interface which facilitates the execution of parallel
algorithms. Functory supports multi-core architectures and distributed
networks; it has also a robust fault-tolerance mechanism.
%
Concerning a distributed implementation, one of the main issues is to
limit the size of data involved in transactions between the master and
the workers. For instance, the size of $\mathcal{V}$ can quickly
become a bottleneck in an architecture based on message passing
communications. As a future work, we plan to develop a distributed
implementation that will only need to send updates of data-structures.

\section{Experimental Results and Future Works}
\label{experiments}

We have evaluated Cubicle on some classic and challenging mutual
exclusion algorithms and cache coherence protocols. In the table
bellow,
%Figure~\ref{fig:benchs}, 
we compare Cubicle's performances with state-of-the-art model checkers
for parameterized systems.  All benchmarks have been executed on a 64
bits machine with a quad-core Intel$^{\textregistered}$
Xeon$^{\textregistered}$ processor @ 3.2 GHz and 24 GB of memory.  For
each tool, we report the results obtained with the best settings we
found. Note that the parallel version of Cubicle was run on 4 cores
and that we only give its results for significantly time consuming
problems. We denote by \textsf{X} benchmarks that we were unable to
translate due to syntactic restrictions.

%\begin{figure}[htb]
%  \centering 
\begin{center}
  \begin{tabular}{|l|c|c|c|c|c|}
    \cline{2-6}
    \multicolumn{1}{l|}{} & \multicolumn{2}{c|}{\textbf{Cubicle}} & ~MCMT~\cite{mcmt}~ &
     ~Undip~\cite{undip}~ & ~PFS~\cite{pfs}~ \\
    \multicolumn{1}{l|}{} & seq & 4 cores & & & \\
    \hline
    \textsf{bakery} & 0.01s & - & 0.01s & 0.04s & 0.01s \\
%    \hline
%    \textsf{Dekker} & 0.01s & - & & & \\
    \hline
    \textsf{Dijkstra} & 0.24s & - & 0.99s & \textbf{0.04s} & 0.26s \\
    \hline
    \textsf{Distributed\_Lamport} & \textbf{2.3s} & - & 12.7s & unsafe & \textsf{X} \\
    \hline
    \textsf{Java\_Mlock} & 0.04s & - & 0.06s & 0.25s & 0.02s \\
%    \hline
%    \textsf{mux\_sem} & \textbf{0.01s} & - & \textbf{0.01s} & & \\
    \hline
    \textsf{Ricart\_Agrawala} & \textbf{1.8s} & - & 1m12s & 4.3s & \textsf{X} \\
    \hline
    \textsf{Szymanski\_at} & \textbf{0.12s} & - & 0.71s & 13.5s  & timeout \\
%    \hline
%    \textsf{two-semaphores} & \textbf{0.01s} & - & \textbf{0.01s} & & \\
    \hline \hline
    \textsf{Berkeley} & 0.01s & - & 0.01s & 0.01s & 0.01s \\
     \hline
    \textsf{flash\_aggregated~\cite{flash_pvs}}& 0.01s & - & 0.02s & 0.01s & \textsf{X} \\
    \hline
    \textsf{German\_Baukus}& \textbf{25.0s} & 17.1s & 3h39m & 9m43s & \textsf{X} \\
    \hline
    \textsf{German\_pfs}& \textbf{6m23s} & 3m8s & 11m31s & timeout & 47m22s \\
    \hline
    \textsf{German\_undip} & \textbf{0.17s} & - & 0.57s & 1m32 & \textsf{X} \\
    \hline
    \textsf{Illinois} & 0.02s & - & 0.04s & 0.06s & 0.06s \\
    % \hline
    % \textsf{Mesi} & 0.01s & - & 0.01s & 0.01s & 0.01s \\
    \hline
    \textsf{Moesi} & 0.01s & - & 0.01s & 0.01s & 0.01s \\
    % \hline
    % \textsf{synapse} & 0.01s & - & 0.01s & 0.01s & 0.01s \\
    \hline
  \end{tabular}
\end{center}
%  \caption{Benchmarks}
%  \label{fig:benchs}
%\end{figure}

%   \begin{tabular}{|l|c|c|c|c|c|}
%     \cline{2-6}
%     \multicolumn{1}{l|}{} & \multicolumn{2}{c|}{\textbf{Cubicle}} & ~MCMT~\cite{mcmt}~ &
%     ~PFS~\cite{pfs}~ & ~Undip~\cite{undip}~ \\
%     \multicolumn{1}{l|}{} & seq & 4 cores & & & \\
%     \hline
%     \textsf{Bakery} & \textbf{0.01s} & / & \textbf{0.01s} & \textbf{0.01s} & \\
% %    \hline
% %    \textsf{Dekker} & 0.01s & / & & & \\
%     \hline
%     \textsf{Dijkstra} & \textbf{0.24s} & / & 0.99s & 0.26 s & \\
%     \hline
%     \textsf{Distrib\_Lamport} & ? & / & & & \\
%     \hline
%     \textsf{Java\_Mlock} & \textbf{0.04s} & / & 0.06s & &  \\
% %    \hline
% %    \textsf{mux\_sem} & \textbf{0.01s} & / & \textbf{0.01s} & & \\
%     \hline
%     \textsf{Rickart} & ? & / & & & \\
%     \hline
%     \textsf{Szymanski\_at} & \textbf{0.12s} & / & 0.71s & timeout & \\
% %    \hline
% %    \textsf{two-semaphores} & \textbf{0.01s} & / & \textbf{0.01s} & & \\
%     \hline \hline
%     \textsf{Berkeley} & \textbf{0.01s} & / & \textbf{0.01s} & & \\
%      \hline
%     \textsf{flash\_aggregated~\cite{flash_pvs}}& 0.01s & / & & & \\
%     \hline
%     \textsf{German\_Baukus}& \textbf{25.0s} & 17.4s & 3h39m & & \\
%     \hline
%     \textsf{German\_pfs}& \textbf{6m5s} & 3m12s & 11m31s & 47m22s & \\
%     \hline
%     \textsf{German\_undip} & 0.17s & / & & & 1m32 \\
%     \hline
%     \textsf{Illinois} & \textbf{0.02s} & / & 0.04s & & \\
% %    \hline
% %    \textsf{Mesi} & \textbf{0.01s} & / & \textbf{0.01s} & & \\
%     \hline
%     \textsf{Moesi} & \textbf{0.01s} & / & \textbf{0.01s} & & \\
%     \hline
%     \textsf{synapse} & \textbf{0.01s} & / & \textbf{0.01s} & & \\
%     \hline
%  \end{tabular}
%  \caption{Benchmarks}
%  \label{fig:benchs}
%\end{figure}


Our experiments are very promising. They show first that the
sequential version of Cubicle is competitive. The parallel version on
4 cores achieves speedups of $1.8$ approximately, which is a good
result considering the fact that cores cannot be fully exploited
because of the synchronization required to perform a pertinent search.
%
In practice, we found that the best setting for Cubicle is to use all
the optimizations described in Section~\ref{sec:optims} (except for
invariant synthesis which can be time consuming). In the table bellow,
%Figure~\ref{fig:german}, 
we show the respective effect of these optimizations on the version of
the German protocol from~\cite{baukus} (\textsf{German\_baukus}). In
particular, it is worth noting that the subtyping analysis increases
performances by an order of magnitude on this example.

\medskip
\begin{center}
  \begin{tabular}{|c|c|c||c|c|}
    \hline
    \multicolumn{3}{|c||}{Optimizations} &
    \multicolumn{2}{c|}{\textbf{Real Time} (\# nodes)}\\
    
    delete nodes & subtyping & invariant generation
    & sequential & 4 cores \\
    \hline
 %    \textsf{german\_pfs} 
%     & No  & 1 & / & No  & 3 & 19217 &  18m0s  \\
%     \cline{2-8}
%     & Yes & 1 & / & No  & 3 & 15737 &  11m31s \\
%     \cline{2-8}
%     & Yes & 1 & / & Yes (110) & 3 & 8004  &  \textbf{6m5s}  \\
%     \cline{2-8}
%     & Yes & \textbf{8} & / & No  & 3 &  15517 & 4m17s  \\
%     \cline{2-8}
%     & Yes & \textbf{8} & / & Yes (169) & 3 & 8031  & \textbf{2m5s}  \\
% 
    No & No & No  & 50m8s (22580) & 27m13s (20710) \\
    \hline    
    Yes & No & No  & 35m16s (20405) & 19m39s (19685) \\ %14m36s (20179)\\
    \hline
    Yes & No & Yes  & 20m45s (15089) & 13m55s (14527) \\
    \hline
    Yes & Yes & No  & \textbf{25.0s} (3322) & \textbf{17.1s} (3188) \\
    % \cline{2-8}
    % & Yes & 1 & Yes & Yes (40) & 3 & 2892 & 51.4s  \\
    % \cline{2-8}

    % \cline{2-8}
    % & Yes & \textbf{8} & Yes & Yes (46) & 3 & 2899 & 20.3s  \\
    \hline
  \end{tabular}
\end{center}
\medskip

% \begin{figure}[htb]
%   \centering 
%   \begin{tabular}{|c|c|c||c|c|}
%     \hline
%     \multicolumn{3}{|c||}{Optimizations} &
%     \multicolumn{2}{c|}{\textbf{Real Time} (\# nodes)}\\
    
%     delete nodes & subtyping & invariant generation
%     & sequential & 4 cores \\
%     \hline
%  %    \textsf{german\_pfs} 
% %     & No  & 1 & / & No  & 3 & 19217 &  18m0s  \\
% %     \cline{2-8}
% %     & Yes & 1 & / & No  & 3 & 15737 &  11m31s \\
% %     \cline{2-8}
% %     & Yes & 1 & / & Yes (110) & 3 & 8004  &  \textbf{6m5s}  \\
% %     \cline{2-8}
% %     & Yes & \textbf{8} & / & No  & 3 &  15517 & 4m17s  \\
% %     \cline{2-8}
% %     & Yes & \textbf{8} & / & Yes (169) & 3 & 8031  & \textbf{2m5s}  \\
% % 
%     No & No & No  & 50m8s (22580) & 27m13s (20710) \\
%     \hline    
%     Yes & No & No  & 35m16s (20405) & 19m39s (19685) \\ %14m36s (20179)\\
%     \hline
%     Yes & No & Yes  & 20m9s (15089) & 14m4s (14516) \\
%     \hline
%     Yes & Yes & No  & \textbf{25.0s} (3322) & \textbf{17.4s} (3159) \\
%     % \cline{2-8}
%     % & Yes & 1 & Yes & Yes (40) & 3 & 2892 & 51.4s  \\
%     % \cline{2-8}

%     % \cline{2-8}
%     % & Yes & \textbf{8} & Yes & Yes (46) & 3 & 2899 & 20.3s  \\

%     \hline
%   \end{tabular}
%   \caption{Cubicle's optimizations on the
%     \textsf{German\_baukus} protocol}
%   \label{fig:german}
% \end{figure}

% \begin{itemize}

% \item Comparison between seq and concurrent bench with our different
% strategies (killed geninv?)

% \item comparison with MCMT (others?) on the german (and its variants?)

% \item other benchs?
% \end{itemize}

As future work, we would like to harness the full power of the SMT
solver by sharing its data structures and even more tightly
integrating its features in the model checker. In particular, this
would be very useful to discover symmetries and to simplify nodes by
finding semantic redundancies modulo theories.  We are also interested
in exploiting the unsat cores returned by the solver to improve our
node deletion mechanism.

\bibliographystyle{abbrv}
\bibliography{tool}

\end{document}
