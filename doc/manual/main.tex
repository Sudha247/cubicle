
\documentclass[a4paper,12pt]{article}
\usepackage[T1]{fontenc}
\usepackage[latin1]{inputenc}
\usepackage{amssymb}
\usepackage{graphicx}

\newcommand{\version}{0.1}

\title{\textbf{\Huge Cubicle}\\\Large Version \version}
\author{{\Large Sylvain Conchon} \\[1em] Universit\'e Paris-Sud 11}
\date\today


\begin{document}

\maketitle

\newpage

\section{Introduction}

Cubicle is a symbolic model checker for array-based systems. It
combines a model checking algorithm and an SMT solver.

\section{Getting started}

Cubicle is free software distributed under the terms of the Apache
Software License version 2.0. You can download it at:

\begin{center}
  \texttt{http://cubicle.lri.fr/}
\end{center}

To compile Cubicle from its source code, you need a version of the
Ocaml compiler\footnote{\texttt{http://caml.inria.fr}} greater that
3.11 and a version of the Alt-Ergo theorem
prover\footnote{\texttt{http://alt-ergo.lri.fr}} greater that
0.94. Then, you just need to:

\begin{enumerate}
  \item run \texttt{autoconf}
  \item configure with \texttt{./configure}
  \item compile with \texttt{make}
  \item install with \texttt{make install} (you may need superuser
    permissions for this last step)
\end{enumerate}

The use of Cubicle is simple : just write a file with the
\texttt{.cub} extension containing the description of a protocol and
run cubicle on that file. The result is a message indicating whether
the protocol is \emph{safe} or \emph{unsafe}.

As an introduction to the syntax of Cubicle's input language, we show
how to write a simple version of Lamport's Bakery algorithm as
described in~\cite{Abdulla}.

First, we define a type \texttt{state}:
\begin{verbatim}
type state = Idle | Wait | Crit | Crash
\end{verbatim}
%
This enumeration type \texttt{state} contains four elements
\texttt{Idle}, \texttt{Wait}, \texttt{Crit} and \texttt{Crash} which
represents the different states of a process.

\begin{verbatim}

\end{verbatim}

\section{Input Language}

The input language of Cubicle is 

\paragraph{Enumeration Types}

\begin{verbatim}
type location = L1 | L2 | L3
\end{verbatim}

\paragraph{Global variables and arrays}

\paragraph{Transitions}

\paragraph{Universal guards}

\section{Examples}

\subsection{The Dekker algorithm}

\begin{verbatim}
globals = Turn[proc]
arrays = Want[proc,bool] Crit[proc,bool]

init (z) { Want[z] = False && Crit[z] = False }

unsafe (z1 z2) { Crit[z1] = True && Crit[z2] = True }

transition req (i)
require { Want[i] = False }
Want[j] := {| i=j : True | _ : Want[j] }

transition enter (i)
require { Want[i]=True && Crit[i] = False && Turn = i}
Crit[j] := {| i=j : True | _ : Crit[j]}

transition exit (i)
require { Crit[i] = True}
assign { Turn := . }
Crit[j] := {| i=j : False | _ : Crit[j] }
Want[j] := {| i=j : False | _ : Want[j] }
\end{verbatim}


\subsection{A simplify version of German protocol}

\begin{verbatim}
type req = Empty | Reqs | Reqe
type cstate = Invalid | Shared | Exclusive

globals = Exgntd[bool] Curcmd[req] Curptr[proc]
arrays = Cache[proc,cstate] Shrset[proc,bool]

init (z) { Cache[z] = Invalid && Shrset[z] = False &&
           Exgntd = False && Curcmd = Empty }

unsafe (z1 z2) { Cache[z1] = Exclusive && Cache[z2] = Shared }

transition req_shared (n)
require { Curcmd = Empty && Cache[n] = Invalid }
assign { Curcmd := Reqs; Curptr := n }
    
transition req_exclusive (n)
require { Curcmd = Empty && Cache[n] <> Exclusive }
assign { Curcmd := Reqe; Curptr := n }
    
transition inv_1 (n)
require { Shrset[n]=True  &&  Curcmd = Reqe }
assign { Exgntd := False }
Cache[j] := {| j = n : Invalid | _ : Cache[j] }
Shrset[j] := {| j= n : False | _ : Shrset[j] }

transition inv_2 (n)
require { Shrset[n]=True  && Curcmd = Reqs && Exgntd=True }
assign { Exgntd := False }
Cache[j] := {| j = n : Invalid | _ : Cache[j] }
Shrset[j] := {| j= n : False | _ : Shrset[j] }
    
transition gnt_shared (n)
require { Curptr = n && Curcmd = Reqs && Exgntd = False }
assign { Curcmd := Empty }
Shrset[j] := {| j = n : True | _ : Shrset[j] }
Cache[j] := {| j = n : Shared | _ : Cache[j] }

transition gnt_exclusive (n)
require { Curcmd = Reqe && Exgntd = False && Curptr = n}
uguard (l) { Shrset[l] = False }
assign { Curcmd := Empty; Exgntd := True }
Shrset[j] := {| j = n : True | _ : Shrset[j] }
Cache[j] := {| j = n : Exclusive | _ : Cache[j] }
\end{verbatim}

\end{document}
