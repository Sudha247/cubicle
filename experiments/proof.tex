\documentclass{article}

\usepackage[english]{babel}
\usepackage[utf8]{inputenc}
\usepackage{verbatim}
\usepackage{alltt}
\usepackage[bookmarks=false]{hyperref}
\usepackage{amsmath, amsthm, amssymb, amsfonts}
\usepackage{graphicx}
\usepackage{tikz}

\newtheorem{thm}{Theorem}
\newtheorem{cor}[thm]{Corollary}

\begin{document}


\section{States closed by subset}

\begin{figure}[h]\centering
  \includegraphics[height=0.3\textwidth]{incl2.mps}
  \caption{Fixpoint by inclusion}
\label{fig:incl}
\end{figure}

\begin{thm}
  \label{thm1}
  \begin{equation*}
    \label{eq:1}
    \forall p . (p \subseteq pre(s) \implies 
    p \subseteq pre(s \setminus (pre(s) \setminus p)))
  \end{equation*}
\end{thm}

\paragraph{Properties of $pre$}

\begin{equation}
  \label{eq:pre1}
  \forall s . pre(s \cap pre(s)) = s \cap pre(s)
\end{equation}

\begin{equation}
  \label{eq:pre2}
  \forall X,Y . pre(X) = X \implies pre(X \cap Y) = X \cap Y
\end{equation}

\begin{equation}
  \label{eq:pre3}
  \forall X,Y . pre(X) = X \implies pre(X \cup Y) = X \cup pre(Y)
\end{equation}


\begin{proof}
Let $p$ such that $p \subseteq pre(s)$,
\[
\begin{array}{rl}
  \mbox{we have } & s \setminus (pre(s) \setminus p) = 
  (s \cap pre(s) \cap p) \cup (s \setminus pre(s))\\
  \mbox{and } &s = (s \cap pre(s)) \cup (s \setminus pre(s))\\
  &pre(s) = pre((s \cap pre(s)) \cup (s \setminus pre(s)))\\
  \mbox{so, }& p \subseteq pre((s \cap pre(s)) \cup (s \setminus
  pre(s))) \\
  &p \subseteq (s \cap pre(s)) \cup pre(s \setminus pre(s))
  ~~~~\mbox{(\ref{eq:pre1} + \ref{eq:pre3})}\\
  &p \subseteq ((s \cap pre(s)) \cup pre(s \setminus pre(s))) \cap p\\
  \mbox{i.e. }&p \subseteq (s \cap pre(s) \cap p) \cup
  pre(s \setminus pre(s))\\
  \mbox{(\ref{eq:pre1} + \ref{eq:pre2}) gives us }& s \cap pre(s) \cap
  p = pre(s \cap pre(s) \cap p)\\
  \mbox{by (\ref{eq:pre3}), }& p \subseteq pre((s \cap pre(s) \cap p) \cup
  (s \setminus pre(s)))\\
\end{array}
\]
\end{proof}

\clearpage

\section{States closed by fixpoint}

\begin{figure}[h]\centering
  \includegraphics[height=0.3\textwidth]{impl.mps}
  \caption{Fixpoint by implication}
\label{fig:impl}
\end{figure}


\begin{cor}
  Let $P = \bigvee_{p \in nodes}{p}$ and
  let $I$ be a Craig interpolant of $(pre(s), \neg P)$ such that $I
  \subseteq pre(s)$,
  \[
  \begin{array}{rl}
    \mbox{If}& pre(s) \implies P\\
    \mbox{then}& pre(s \setminus (pre(s) \setminus I)) \implies P
  \end{array}
  \]
\end{cor}


\begin{proof}

  By Theorem \ref{thm1}, $I \subseteq pre(s \setminus (pre(s) \setminus I))$,
  so $pre(s \setminus (pre(s) \setminus I)) \implies I \implies P$. 

\end{proof}

\paragraph{Remark.} 
% $I$ is an \emph{under-approximation} of the state(s) that is (are)
% backward reachable from $s$ by transition $\tau$.\\ 
Intuitionally, $I$ represents the part of $pre(s)$ that is needed to
obtain a fixpoint.

\paragraph{TODO.} 
Prove that such an $I$ exists.

\paragraph{Example.} 
Let sum type \texttt{t = A | B}, suppose we
know that the state $p = \texttt{\{Y = A\}}$ is a fixpoint.\\
\[
\mbox{transition }\tau : \left\{
\begin{array}{l}
  \mbox{require }\{ \texttt{Y $\neq$ B} \}\\
  \texttt{Y := B}\\
  \texttt{Z := False}\\
\end{array}\right.
\]

Let $s = \{ \Gamma, \texttt{Y = B} \}$ with $(\texttt{Z := False}),
(\texttt{Y := B}) \not \in \Gamma$.

\[
pre_\tau(s) = \{ \Gamma, \texttt{Y $\neq$ B} \}
\]

Clearly $pre_\tau(s) \implies p$, let $I = \{\texttt{Y $\neq$
  B}\}$. $pre_\tau(s)$ implies $I$ which in turn implies $p$, so $I$
is an interpolant and $I$ is a subset of $pre_\tau(s)$.

\[
s \setminus (pre_\tau(s) \setminus I) = s \setminus \Gamma =
\{\texttt{Y = B}\}
\]

\[
pre_\tau(s \setminus (pre_\tau(s) \setminus I)) = \{\texttt{Y $\neq$
  B}\} \implies p
\]

\end{document}
